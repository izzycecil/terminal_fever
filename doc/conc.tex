In this report we have proposed a collection of enhancements that
could be made to the typical Unix command line shell. We've suggested these
changes in order to tackle the shell's most glaring issues: the steep learning
curve, the dearth of contextual information, and the ease with which one can
irrecoverably lose information. Targeting those learning Unix for the first
time, we hoped to demonstrate that bringing common GUI elements to the command
line could drastically improve the user experience. Once we've left the
prototype phase a more extensive analysis can be carried out, as described in
Section \ref{sec:eval}.

Thus far we have designed a basic web-based prototype of the system, one that
highlights it's novel GUI techniques and core functionality. As is discussed
above, we've gone through a preliminary user study of our prototype. This study
consisted of an entry survey to gauge users' prior experience with the shell,
and attitudes towards the shell. Following a walkthrough of the prototype we
queried users on their experience: whether they found it usable, educational,
and enjoyable. These results can now be used to evaluate and refine the
prototype in response to users' critiques.

Based upon our analysis of the user surveys, we conclude that our hypothesis in
regards to usability is not supported with this version of our prototype. The
command line interface certainly can be made more user friendly by the
application of user interface design techniques, and more work and research on
our prototype can help us accomplish that goal in the future.

% \emph{THE END}

% From the responses we can speculate a bit as to why, and how we can improve
% future versions of the application.

% * users did not find the design compelling
% * users found the interface confusing
% * users %...

% 9a. Briefly describe what the system is and how you evaluate the system
% 9b. Use a list to describe which hypotheses in Section 7 are supported in the evaluation
% 9c. (optional) List the findings which are in the hyphteses.
% 9d. (optional) List possible strategies to enhanced the system .
%%% Local Variables:
%%% mode: latex
%%% TeX-master: "documentation"
%%% End:
