Throughout this report, we have demonstrated a collection of enhancements that
could be made to the typical Unix command line shell in order to increase
efficiency, usability, and understand-ability of the system. Targeting those
learning Unix for the first time, we hoped to demonstrate ways to help teach
beginners, by bringing common GUI techniques to the command line. Once we have
left the prototype phase, a more extensive analysis can be carried out, as
described in Section \ref{sec:eval}.

Thus far we have designed a basic web-based prototype of the system, one that
highlights it's novel GUI techniques and core functionality. As is discussed
above, we've gone through a preliminary user study of our prototype. This study
consisted of an entry survey to gauge users' prior experience with the shell,
and attitudes towards the shell. Following a walkthrough of the prototype, we
queried users on their experience: whether they found it usable, educational,
and enjoyable. These results can now be used to evaluate and refine the
prototype in response to users' critiques.

In Section \ref{sec:results} we analyze the user responses to both surveys. This
section of testing was intended to test the hypothesis that users who use our
program will report a more positive user experience. On the entry survey
querying a user's self-reported experience with the shell with regards to
usability, efficiency, and knowledge users reported an average rating of 1.9,
2.2, and 2.0 respectively (from 1 to 4 in a Likert-style questionnaire). On the
exit survey users ranked the prototype a 1.9, 1.7, and 1.9 on its usability,
educational potential, and their experience with it respectively (from 1 to 4 in
a Likert-style questionnaire).

Based upon these ratings, we conclude that our hypothesis here is not shown by
this prototype. From the responses we can speculate a bit as to why.

* users did not find the design compelling
* users found the interface confusing
* users ...

9a. Briefly describe what the system is and how you evaluate the system
9b. Use a list to describe which hypotheses in Section 7 are supported in the evaluation
9c. (optional) List the findings which are in the hyphteses.
9d. (optional) List possible strategies to enhanced the system
.
%%% Local Variables:
%%% mode: latex
%%% TeX-master: "documentation"
%%% End:
