\begin{figure}[h]
  \centering
  \includegraphics[width=0.85\linewidth]{figures/sketch.png}
  \caption{A sketch of our problem design.}
  \label{fig:sketch}
\end{figure}

Shown in Figure (\ref{fig:sketch}) is the problem sketch for our project. The
frames are color-coded according to the model-view-controller (MVC) architecture
design pattern: green for the model; blue for the view; red for the controller.

The sketch begins with a user, of any experience level, who interacts with the
system via keyboard and mouse. The main user interface of the system is an
ordinary command line window, which will read and execute commands. In addition,
our program will have an adjacent contextual information window, which provides
heads up information about the state of the system, and can preview the results
of commands before they are run.

A user's input in passed from the interface through the interaction engine. The
interaction engine processes this text for command suggestions, autocompletion,
and automation, returning this information to the command line in the interface.
The OS's man pages, this package's built-in patterns, the user's history, and
the current state of the shell are all used by the interaction engine for these
features.

This system has a two-step state design, consisting of an ``active'' state and a
``preview'' state. For any command run, changes to the file system or to the
shell state are staged, or saved, in the preview state. If the user approves of
whatever changes were made, then the preview state is flushed to the active
state, which is synonymous with the actual files on disk and the actual shell
environment variables.
