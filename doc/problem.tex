\begin{figure}[h]
  \centering
  \includegraphics[width=0.85\linewidth]{figures/sketch.png}
  \caption{A sketch of our problem design.}
  \label{fig:sketch}
\end{figure}

Shown in Figure (\ref{fig:sketch}) is the problem sketch for our project. The
frames are color-coded according to the model-view-controller (MVC) architecture
design pattern: green for the model; blue for the view; red for the controller.

The sketch begins with a user, of any experience level, who interacts with the
system via keyboard and mouse. The main user interface of the system is an
ordinary command line window, which will read and execute commands. In addition,
our program will have an adjacent contextual information window, which provides
heads up information about the state of the system, and can preview the results
of commands before they are run.

A user's input in passed from the interface to an interaction engine. The
interaction engine processes this text for command suggestions, autocompletion,
and automation, returning this information to the command line interface. The
OS's man pages, the package's built-in patterns, the user's history, and the
current state of the shell are all used by the interaction engine to
intelligently provide these features.

This system uses a two-step design to model the changes that the command line
applies to the operating system. This allows us to provide an ``escape hatch''
before running commands, which is important because it's possible for users to
make destructive mistakes when using the command line! Every time a command is
run, the changes to the filesystem or OS state are first staged in the ``preview
state.'' The staged changes are visible to the user through the context window,
where they can be approved or rejected. If the user approves of the action, then
the preview state is flushed and the changes are sent to the ``active state,''
which is synonymous with the actual underlying operating system.