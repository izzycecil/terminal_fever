\begin{figure}[h]
  \centering
  \includegraphics[width=0.8\linewidth]{figures/sketch.png}
  \caption{A sketch of our problem design.}
  \label{fig:sketch}
\end{figure}

Shown in Figure (\ref{fig:sketch}) is the problem sketch for our project. The
frames are color-coded according to the model-view-controller (MVC) design
pattern: green is model; blue is view; red is controller.

The sketch begins with a user, of any experience level, who interacts with the
system via keyboard and mouse. Their view into the system consists of a CLI
window and a contextual information window.

A user's input in passed from the interface through the interaction engine. The
interaction engine processes this text for command suggestions, autocompletion,
and automation. The OS's man pages, predefined built-in patterns, the user's
history, and the current state of the shell are all used by the interaction
engine for these features.

This system has a two-step state design, consisting of an 'active' state and a
'preview' state. For any command run, changes to the file system or to the state
of the shell are staged - saved in the preview state. If the user approves of
whatever changes were made, then the preview state is flushed to the active
state, and files on disk or environment variables are updated accordingly.
