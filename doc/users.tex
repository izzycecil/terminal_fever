\subsection{Beginners}
Our first imagined user is brand new to the shell, if not Linux itself. A
freshman CSE student, for example, would need to learn how to navigate a UNIX
operating system via its main interface - the shell. As a first-time Linux/BSD
user they would need to learn some basic information about how their system is
organized, and how to manipulate it. From their previous computer usage they
could be expected to understand the hierarchical nature of file systems, but
nothing more. What this user needs is to reach a baseline fluency in CLI usage:
how to navigate their file system; how to create and modify files; how to learn
more.
\begin{enumerate}
\item No experience with UNIX operating systems
\item No experience with a CLI
\item No experience with the state inherent to a shell, mainly the current working directory
\item No knowledge of what programming is, or how to automate tasks
\item No knowledge of the anatomy of a command (command name, arguments, options, et cetera)
\item No knowledge of the software available on their system
\item No knowledge of basic OS tasks, such as installing software
\item Has experience with common GUI techniques - click-to-open, drag-and-drop, et cetera
\item Is uncertain
\end{enumerate}

\subsection{Casual Users}
Another potential user would be for scientists, engineers, or anyone with
experience as a technical computer user whose workflow could be more efficient
if they knew how to use the shell. The CLI can make it easier for users to
automate tasks and make precise changes to the files they are working on. These
users may already know how to complicated tasks on their computers, or even be
able to write programs. Many powerful tools insulate their users from the
details they would need to use the shell, such as Matlab, Maple, or Excel. An
information-rich user interface could help casual power users make the
transition to understand how to use their terminals for more complicated tasks
such as shell scripting and task automation.

\begin{enumerate}
\item Understands the basics of how computers work and filesystems are organized
\item Understands what computer programs are or has knowledge of how to program
\item Is only familiar with a GUI, or is more comfortable with pointing and clicking than typing
\item Regularly needs to perform some repetitive task on their computer, but does not imagine the task could be automated
\item Finds themselves lost if/when they use the CLI - doesn’t understand the larger context of the system surrounding the shell
\item Is unaware of most of the programs, commands, and command line options they could be using (has only seen the tip of the iceberg)
\item Does not know how to apply technical knowledge or programming skills to system tasks in the CLI
\end{enumerate}
\subsection{Advanced User}
We also believe that advanced users who already know their way around the
command line could benefit from an information-rich shell. This user can be
expected to know by heart the commands they need to use, the details of how
their system is organized, and how to program in a shell language. Where the CLI
could stand improvement for these users is in providing more details on the
state of their system, and allowing a terser search of console-oriented
documentation.

\begin{enumerate}
\item Deep understanding of OS structure and interfaces
\item Mastery of command line usage
\item Commonly constructs aliases and automates tasks
\item Consults system documentation to find advanced command options
\item Understands that task inference is frequently achievable from context in the shell
\end{enumerate}
