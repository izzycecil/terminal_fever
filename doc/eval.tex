Evaluating the effectiveness of our design is an important step in understanding
how we can and have effected users' abilitie with and knowledge of the command
line. An important metric for this would be \emph{unique command count} --- the
measure of how many unique commands the user is able to use. This includes
usage of things such as looping, options, as well as the actual commands. Our
goal is to understand the impact our designs could have on unique command count
and the general user experience.

We hypothesize the following:
\begin{itemize}
  \item Users who use our program will have a higher unique command count than
  the control group.
  \item Users who use our program will report a more positive user experience.
  \item Users who use our program will maintain their unique command count
  outside of our shell.
\end{itemize}
These captor the idea that the users should begin using the shell better, have
less fear of the shell, and will be able to carry this knowledge to other
systems.

\subsection{Collecting Data}
Collecting data to this end will take place in two parts:
\begin{itemize}
  \item A user survey to gauge users emotional experience --- how much they
  enjoyed using the command line.
  \item A data reporting study, which would report command history for a number
  of users.
\end{itemize}
These would each need to be run for our two groups: control and enhanced. Likely
our study would require each group (of mixed experience levels), to use our
software for a period of time. Then we would give them a number of tasks to
perform. After they have completed the tasks, we can collect statistics, and
conduct the survey.

Our user survey would need to focus in on the emotional experience of our
users. It would ask questions such as
\begin{quote}
  \begin{itemize}
    \item Rank your intimidation of the system.
    \item Rank how hard it was to perform a given task.
    \item Rank how well you think you used the system.
    \item Rank how happy you were with your experience.
  \end{itemize}
\end{quote}

Automated usage statistics would be much easier. All we would need to do is get
the history file for each user, and look at the unique command usage.

Together we can use these to test the first two of our hypotheses. For the
third, we would require users to switch from our system to a traditional system
after an extended period. At that point we could likely re-conduct our analysis
to see if users have learned skills in a way they can reproduce.
\begin{figure}[ht]
  \centering
  \includegraphics[width=0.8\linewidth]{figures/user_study.png}
  \caption{\label{fig:ustud} This shows a timeline of how we would evaluate our
    design.}
\end{figure}
Figure \ref{fig:ustud} shows something of a timeline on how this would work to
develop our understanding. Again, in it we have two groups of users. Each
receives training in a different command line system. Each will be given a set
of tasks after some period of time. After these tasks, we can begin to analyze
the results.



%%% Local Variables:
%%% mode: latex
%%% TeX-master: "documentation"
%%% End:
