As this is an addition to the CLI, and not a departure from it, we intend to
keep the keyboard as the main focus. All new interfaces should be navigable by
keyboard, if not least primarily so. For users new to a CLI, however, they can
be assumed to be most comfortable with interacting via a mouse. We imagine that
for the novice and casual user alike, a hybrid interface could be highly useful.

The Gestalt Principles

Proximity --- In the shell there is a fixed amount of space between lines and
individual characters is fixed. However, textual output can still take advantage
of the principle of proximity by using whitespace to separate items that should
not be grouped together. This achieves the goal of grouping related objects in a
text-based interface.

Similarity --- Text mainly conveys semantic information, and we can use a few
techniques to visually tag that information so that the user can easily
determine the “type” of the text they’re reading and writing. Specifically,
automatically making use of output coloring and syntax highlighting can help CLI
a user pick apart information and group similar words within the buffer.

Past Experience --- Most importantly, we need to appeal to the experiences of both
novice and advanced users. Those who have never used the shell will expect their
filesystem to be presented to them in a way that reminds them of a file browser
- which is why we want to include a graphical pane which can be used to fill in
commands equivalent to point and click actions. More advanced users will be more
familiar with the basics of common commands, and can benefit from command
completion that will make it easier to type the things that they type most
often.

Color and light patterns could be used to convey information about the current
session. In class the structure of eyes was discussed, with a focus on what sort
of light is most easily noticed by different sections of they eye. In a
text-based interface this theory could be used to craft notifications that catch
the user’s attention based on the distance of the notification from the user’s
likely focus on the screen.

%%% Local Variables:
%%% mode: latex
%%% TeX-master: t
%%% End:
