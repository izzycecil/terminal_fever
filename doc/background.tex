In the lecture from the first day of our Human Computer Interaction class, we
talked about the history of computer user interfaces. One slide mentioned the
command line interface (CLI) was mentioned, as an example of a historical type
of interface. It was described as ``efficient, precise, and fast.''

Unfortunately, this advanced interface has gained its power at the expense of
ease-of-use. Utilizing the CLI entails a ``large overhead to learning set of
commands,'' which is what has led to peculiar design goals for command line
interfaces. Commands and options are generally terse abbreviations, which are
very easy to memorize, but are not necessarily easy to learn.

As Computer Scientists, the CLI isn’t just relic of history; we make heavy use
it on a daily basis. It’s a powerful and highly flexible tool, useful for a
menagerie of tasks. For the beginner user, the shell can be an incomprehensible
nightmare \--- its blinking cursor against a blackened sky a stand-in for man's
helpless struggle against death. Just as text editors have made leaps and bounds
in communicating essential contextual information, we believe the CLI can be
improved for beginning and advanced users alike.

%%% Local Variables:
%%% mode: latex
%%% TeX-master: "documentation"
%%% End:
